\subsection{数字滤波器的设计}

数字滤波器的设计分为两种,一种是\bd{有限脉冲响应}(FIR)滤波器的设计,
另一种是\bd{无限脉冲响应}(IIR)滤波器的设计。

\subsubsection{低通 FIR 滤波器的设计}

\begin{exercise}
    根据下列指标设计低通 FIR 滤波器,写出其单位冲激响应函数 $h(n)$。
    \begin{figure}[H]
        \centering
        \begin{tabular}{|c|c|}
            \hline
            通带边缘频率:$3.3\;\mathrm{kHz}$ & 过渡带宽度:$4.4\;\mathrm{kHz}$ \\
            \hline
            采样频率:$22\;\mathrm{kHz}$ & 阻带衰减:$43\;\mathrm{dB}$ \\
            \hline
        \end{tabular}
    \end{figure}
    供设计 FIR 滤波器时参考的各种窗函数性能如下图所示(若多种同时满足,则选序号最小的):
    \begin{figure}[H]
        \centering
        \begin{tabular}{|c|c|c|c|c|c|}
            \hline
            \textbf{序号} & \textbf{窗类型} & \textbf{窗函数} & \textbf{窗内项数} & \textbf{阻带衰减} & \textbf{通带边缘增益} \\
            \hline
            1 & 矩形 & $1$ & $0.91 f_s / \text{T.W.}$ & $21$ & $-0.9$ \\
            \hline
            2 & 汉宁 & $0.5 + 0.5\cos(2\pi n / (N-1))$ & $3.32 f_s / \text{T.W.}$ & $44$ & $-0.06$ \\
            \hline
            3 & 哈明 & $0.54 + 0.46\cos(2\pi n / (N-1))$ & $3.44 f_s / \text{T.W.}$ & $55$ & $-0.02$ \\
            \hline
            4 & 布莱克曼 & $0.42 + 0.5\cos(2\pi n / (N-1)) + 0.08\cos(4\pi n / (N-1))$ & $5.98 f_s / \text{T.W.}$ & $75$ & $-0.0014$ \\
            \hline
        \end{tabular}
    \end{figure}
\end{exercise}

\begin{solution}
    截止频率
    \begin{align*}
        f_c = 3.3 + \frac{1}{2} \times 4.4 = 5.5\;\mathrm{kHz},
    \end{align*}
    则
    \begin{align*}
        \omega_c = 2\pi \times \frac{f_c}{f_s} = 2\pi \times \frac{5.5}{22} = \frac{\pi}{2}.
    \end{align*}
    计算得
    \begin{align*}
        h(n) = \frac{\sin(\omega_c n)}{\pi n} = \frac{\sin(\pi n / 2)}{\pi n}.
    \end{align*}
    由阻带衰减为 $43\;\mathrm{dB}$ 知,应当选择汉宁窗。由于
    \begin{align*}
        N \ge 3.32 \times \frac{f_s}{T.W.} = 3.32 \times \frac{22}{4.4} = 16.6,
    \end{align*}
    故取 $N = 17$。则
    \begin{align*}
        w(n) = 0.5 + 0.5\cos\frac{2\pi n}{N - 1} = 0.5 + 0.5\cos\frac{\pi n}{8}.
    \end{align*}
    因此,低通 FIR 滤波器的冲激响应函数为
    \begin{align*}
        h'(n) = h(n) w(n) = \frac{\sin(\pi n / 2)}{\pi n} \left(0.5 + 0.5\cos\frac{\pi n}{8}\right) \quad (\abs{n} \le 8).
    \end{align*}
    将其右移 $(N-1)/2$ 个单位,即得
    \begin{align*}
        h''(n) = \frac{\sin(\pi (n - 8) / 2)}{\pi (n - 8)} \left(0.5 + 0.5\cos\frac{\pi (n - 8)}{8}\right) \quad (0 \le n \le 16).
    \end{align*}
\end{solution}



\subsubsection{带通 FIR 滤波器的设计}


\subsubsection{高通 FIR 滤波器的设计}

\begin{exercise}
    根据下列指标设计高通 FIR 滤波器,写出其单位冲激响应函数 $h(n)$。
    \begin{figure}[H]
        \centering
        \begin{tabular}{|c|c|}
            \hline
            通带边缘频率:$8.8\;\mathrm{kHz}$ & 过渡带宽度:$4.4\;\mathrm{kHz}$ \\
            \hline
            采样频率:$22\;\mathrm{kHz}$ & 阻带衰减:$35\;\mathrm{dB}$ \\
            \hline
        \end{tabular}
    \end{figure}
    供设计 FIR 滤波器时参考的各种窗函数性能如下图所示(若多种同时满足,则选序号最小的):
    \begin{figure}[H]
        \centering
        \begin{tabular}{|c|c|c|c|c|c|}
            \hline
            \textbf{序号} & \textbf{窗类型} & \textbf{窗函数} & \textbf{窗内项数} & \textbf{阻带衰减} & \textbf{通带边缘增益} \\
            \hline
            1 & 矩形 & $1$ & $0.91 f_s / \text{T.W.}$ & $21$ & $-0.9$ \\
            \hline
            2 & 汉宁 & $0.5 + 0.5\cos(2\pi n / (N-1))$ & $3.32 f_s / \text{T.W.}$ & $44$ & $-0.06$ \\
            \hline
            3 & 哈明 & $0.54 + 0.46\cos(2\pi n / (N-1))$ & $3.44 f_s / \text{T.W.}$ & $55$ & $-0.02$ \\
            \hline
            4 & 布莱克曼 & $0.42 + 0.5\cos(2\pi n / (N-1)) + 0.08\cos(4\pi n / (N-1))$ & $5.98 f_s / \text{T.W.}$ & $75$ & $-0.0014$ \\
            \hline
        \end{tabular}
    \end{figure}
\end{exercise}

\begin{solution}
    首先考虑其对应的低通 FIR 滤波器,截止频率
    \begin{align*}
        f_c = 11 - 8.8 + \frac{1}{2} \times 4.4 = 4.4\;\mathrm{kHz},
    \end{align*}
    则
    \begin{align*}
        \omega_c = 2\pi \times \frac{f_c}{f_s} = 2\pi \times \frac{4.4}{22} = \frac{2\pi}{5}.
    \end{align*}
    计算得
    \begin{align*}
        h(n) = \frac{\sin(\omega_c n)}{\pi n} = \frac{\sin(2\pi n / 5)}{\pi n}.
    \end{align*}
    由阻带衰减为 $35\;\mathrm{dB}$ 知,应当选择汉宁窗。由于
    \begin{align*}
        N \ge 3.32 \times \frac{f_s}{T.W.} = 3.32 \times \frac{22}{4.4} = 16.6,
    \end{align*}
    故取 $N = 17$。则
    \begin{align*}
        w(n) = 0.5 + 0.5\cos\frac{2\pi n}{N - 1} = 0.5 + 0.5\cos\frac{\pi n}{8}.
    \end{align*}
    因此,高通 FIR 滤波器的冲激响应函数为
    \begin{align*}
        h'(n) = h(n) w(n) \cos\pi n = \frac{\sin(2\pi n / 5)}{\pi n} \left(0.5 + 0.5\cos\frac{\pi n}{8}\right) \cos\pi n \quad (\abs{n} \le 8).
    \end{align*}
    将其右移 $(N-1)/2$ 个单位,即得
    \begin{align*}
        h''(n) = \frac{\sin(2\pi (n - 8) / 5)}{\pi (n - 8)} \left(0.5 + 0.5\cos\frac{\pi (n - 8)}{8}\right) \cos\pi (n - 8) \quad (0 \le n \le 16).
    \end{align*}
\end{solution}
