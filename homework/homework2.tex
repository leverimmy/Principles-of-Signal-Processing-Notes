\subsection{作业 2}

\begin{homework}
    证明性质 \ref{property:convolution-differential}。
    卷积的微分满足以下性质:两个信号卷积的微分等于其中任一信号的微分与另一信号的卷积,即
    \begin{align*}
        \frac{\D}{\D{t}}\left[f_1(t) * f_2(t)\right]
        = f_1(t) * \frac{\D}{\D{t}}\left[f_2(t)\right]
        = \frac{\D}{\D{t}}\left[f_1(t)\right] * f_2(t),
    \end{align*}
    其中 $f_1, f_2$ 为  $\set{R}$ 上连续可导函数。
\end{homework}

\begin{proof}
    \begin{align*}
        \frac{\D}{\D{t}}\left[f_1(t) * f_2(t)\right]
        & = \frac{\D}{\D{t}}\left[\int_{-\infty}^{+\infty}f_1(a) \cdot f_2(t - a)\D{a}\right] \\
        & = \int_{-\infty}^{+\infty}f_1(a) \cdot \frac{\D}{\D{t}}\left[f_2(t - a)\right]\D{a}.
    \end{align*}
    记 $g(t) = \frac{\D}{\D{t}}\left[f_2(t)\right]$,则 $g(t - a) = \frac{\D}{\D{t}}\left[f_2(t - a)\right]$。因此
    \begin{align*}
        \int_{-\infty}^{+\infty}f_1(a) \cdot \frac{\D}{\D{t}}\left[f_2(t - a)\right]\D{a}
        = \int_{-\infty}^{+\infty}f_1(a) \cdot g(t - a)\D{a}
        = f_1(t) * g(t)
        = f_1(t) * \frac{\D}{\D{t}}\left[f_2(t)\right].
    \end{align*}
    同理,由卷积运算的交换律可以证明 $\frac{\D}{\D{t}}\left[f_1(t) * f_2(t)\right] = \frac{\D}{\D{t}}\left[f_1(t)\right] * f_2(t)$。
    因此,命题得证。
\end{proof}

\begin{homework}
    证明性质 \ref{property:convolution-integral}。
    卷积的积分满足以下性质:两个信号卷积的积分等于其中任一信号的积分与另一信号的卷积,即
    \begin{align*}
        \int_{-\infty}^{t}(f_1 * f_2)(\lambda)\D{\lambda}
        = f_1(t) * \left(\int_{-\infty}^{t}f_2(\lambda)\D{\lambda}\right)
        = \left(\int_{-\infty}^{t}f_1(\lambda)\D{\lambda}\right) * f_2(t),
    \end{align*}
    其中 $f_1, f_2$ 为  $\set{R}$ 上连续可导函数。
\end{homework}

\begin{proof}
    \begin{align*}
        \int_{-\infty}^{t}(f_1 * f_2)(\lambda)\D{\lambda}
        & = \int_{-\infty}^{t}\left[\int_{-\infty}^{+\infty}f_1(a)f_2(\lambda - a)\D{a}\right]\D{\lambda} \\
        & = \int_{-\infty}^{+\infty}f_1(a)\left[\int_{-\infty}^{t}f_2(\lambda - a)\D{\lambda}\right]\D{a}.
    \end{align*}
    记 $g(t) = \int_{-\infty}^{t}f_2(\lambda)\D{\lambda}$,
    则 $g(t - a) = \int_{-\infty}^{t - a}f_2(\lambda')\D{\lambda'} = \int_{-\infty}^{t}f_2(\lambda - a)\D{\lambda}, \lambda = \lambda' + a$。因此
    \begin{align*}
        \int_{-\infty}^{+\infty}f_1(a)\left[\int_{-\infty}^{t}f_2(\lambda - a)\D{\lambda}\right]\D{a}
        = \int_{-\infty}^{+\infty}f_1(a)g(t - a)\D{a}
        = f_1(t) * g(t)
        = f_1(t) * \left(\int_{-\infty}^{t}f_2(\lambda)\D{\lambda}\right).
    \end{align*}
    同理,由卷积运算的交换律可以证明 $\int_{-\infty}^{t}(f_1 * f_2)(\lambda)\D{\lambda} = \left(\int_{-\infty}^{t}f_1(\lambda)\D{\lambda}\right) * f_2(t)$。
    因此,命题得证。
\end{proof}

\begin{homework}
    证明:一个函数与单位阶跃函数的卷积等于该函数的积分,即
    \begin{align*}
        f(t) * u(t) = \int_{-\infty}^{t}f(\tau)\D{\tau}.
    \end{align*}
\end{homework}

\begin{proof}
    \begin{align*}
        f(t) * u(t) & = \int_{-\infty}^{+\infty}f(\tau)u(t - \tau)\D{\tau} \\
        & = \int_{-\infty}^{t}f(\tau)u(t - \tau)\D{\tau} + \int_{t}^{+\infty}f(\tau)u(t - \tau)\D{\tau} \\
        & = \int_{-\infty}^{t}f(\tau)\D{\tau} + 0 \\
        & = \int_{-\infty}^{t}f(\tau)\D{\tau}.
    \end{align*}
    命题得证。
\end{proof}
