\subsection{作业 2}

\begin{homework}
    证明性质 \ref{property:convolution-differential}。
    卷积的微分满足以下性质:两个信号卷积的微分等于其中任一信号的微分与另一信号的卷积,即
    \begin{align*}
        \frac{\D}{\D{t}}\left[f_1(t) * f_2(t)\right]
        = f_1(t) * \frac{\D}{\D{t}}\left[f_2(t)\right]
        = \frac{\D}{\D{t}}\left[f_1(t)\right] * f_2(t),
    \end{align*}
    其中 $f_1, f_2$ 为  $\set{R}$ 上连续可导函数。
\end{homework}

\begin{homework}
    证明性质 \ref{property:convolution-integral}。
    卷积的积分满足以下性质:两个信号卷积的积分等于其中任一信号的积分与另一信号的卷积,即
    \begin{align*}
        \int_{-\infty}^{t}(f_1 * f_2)(\lambda)\D{\lambda}
        = f_1(t) * \left(\int_{-\infty}^{t}f_2(\lambda)\D{\lambda}\right)
        = \left(\int_{-\infty}^{t}f_1(\lambda)\D{\lambda}\right) * f_2(t),
    \end{align*}
    其中 $f_1, f_2$ 为  $\set{R}$ 上连续可导函数。
\end{homework}

\begin{homework}
    证明:一个函数与单位阶跃函数的卷积等于该函数的积分,即
    \begin{align*}
        f(t) * u(t) = \int_{-\infty}^{t}f(\tau)\D{\tau}.
    \end{align*}
\end{homework}
