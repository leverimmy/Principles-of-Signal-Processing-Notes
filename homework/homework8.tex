\subsection{作业 8}

\begin{homework}
    利用 DFT 的公式计算
    \begin{align*}
        x(n) = \sin\left(\frac{2\pi}{N}mn\right)
    \end{align*}
    的 $N$ 点 DFT,其中 $0 \le n \le N - 1, 0 < m < N$ 且 $m \in \set{Z}$。
\end{homework}

\begin{solution}
    由于 $\sin\theta = (\mathe^{\mathi\theta} - \mathe^{-\mathi\theta})/(2\mathi)$,因此
    \begin{align*}
        x(n) & = \sin\left(\frac{2\pi}{N}mn\right) \\
        & = \frac{\mathe^{2\pi\mathi mn/N} - \mathe^{-2\pi\mathi mn/N}}{2\mathi} \\
        & = \frac{\mathi}{2} \left(W_N^{mn} - W_N^{-mn}\right).
    \end{align*}
    由 DFT 的定义可知
    \begin{align*}
        X(k) & = \sum_{n = 0}^{N - 1} x(n)W_N^{kn} \\
        & = \sum_{n = 0}^{N - 1} \sin\left(\frac{2\pi}{N}mn\right)W_N^{kn} \\
        & = \frac{\mathi}{2}\sum_{n = 0}^{N - 1}\left(W_N^{mn} - W_N^{-mn}\right)W_N^{kn} \\
        & = \frac{\mathi}{2}\sum_{n = 0}^{N - 1}W_N^{(k + m)n} - \frac{\mathi}{2}\sum_{n = 0}^{N - 1}W_N^{(k - m)n}.
    \end{align*}
    当 $k + m = N$ 时,$\frac{\mathi}{2}\sum_{n = 0}^{N - 1}W_N^{(k + m)n} = \mathi N / 2$。
    当 $k + m \neq N$ 时,$\frac{\mathi}{2}\sum_{n = 0}^{N - 1}W_N^{(k + m)n}$ 为等比数列求和,故
    \begin{align*}
        \frac{\mathi}{2}\sum_{n = 0}^{N - 1}W_N^{(k + m)n} & = \frac{\mathi}{2}\frac{1 - \left(W_N^{k + mN}\right)^N}{1 - W_N^{k + m}} \\
        & = 0.
    \end{align*}
    当 $k = m$ 时,$\frac{\mathi}{2}\sum_{n = 0}^{N - 1}W_N^{(k - m)n} = \mathi N / 2$。
    当 $k \ne m$ 时,$\frac{\mathi}{2}\sum_{n = 0}^{N - 1}W_N^{(k - m)n}$ 为等比数列求和,故
    \begin{align*}
        \frac{\mathi}{2}\sum_{n = 0}^{N - 1}W_N^{(k - m)n} & = \frac{\mathi}{2} \cdot \frac{1 - \left(W_N^{k - m}\right)^N}{1 - W_N^{k - m}} \\
        & = 0.
    \end{align*}
    综上所述,
    \begin{align*}
        X(k) = \frac{\mathi N}{2}(\delta(k + m - N) - \delta(k - m)).
    \end{align*}
\end{solution}

\begin{homework}
    证明性质 \ref{property:dft_conjugate_symmetry}。
\end{homework}

\begin{proof}
    由于 $W_N = \mathe^{-2\pi\mathi/N}$,因此
    \begin{align*}
        \left(W_N^{(N - k)n}\right)^* & = \left(\mathe^{-2\pi\mathi n(N-k)/N}\right)^* \\
        & = \mathe^{-2\pi\mathi n(k-N)/N} \\
        & = \mathe^{-2\pi\mathi kn/N} \cdot \mathe^{2\pi\mathi n} \\
        & = \left(\mathe^{-2\pi\mathi/N}\right)^{kn} \\
        & = W_N^{kn}.
    \end{align*}
    由于 $x(n)$ 为实序列,故 $x(n) = x^*(n)$,所以由 DFT 的定义可知
    \begin{align*}
        X^*(N - k) & = \left(\sum_{n = 0}^{N - 1} x(n)W_N^{(N - k)n}\right)^* \\
        & = \sum_{n = 0}^{N - 1} x^*(n)\left(W_N^{(N - k)n}\right)^* \\
        & = \sum_{n = 0}^{N - 1} x(n)W_N^{kn} \\
        & = X(k).
    \end{align*}
    命题得证。
\end{proof}

\begin{homework}
    本课介绍了 DFT 的快速算法 FFT。相应地,可将 $N$ 点 IDFT 运算递归地分解
    为 $N/2$ 点,$N/4$ 点,……,$2$ 点 IDFT 运算,从而实现 IDFT 的快速运算。

    设序列 $X(k)$ 的长度为 $N$,$N$ 为
    偶数,$X(k)$ 的 $N$ 点 IDFT 为 $x(n)$。$G(k)$ 是 $X(k)$ 中
    下标为偶数的元素组成的子序列,$H(k)$ 是 $X(k)$ 中下标为奇数的元素
    组成的子序列,它们的长度是 $N/2$,对应的 $N/2$ 点 IDFT 分别
    是 $g(n)$ 和 $h(n)$。试用 $g(n)$ 和 $h(n)$ 表示 $x(n)$。
\end{homework}

\begin{solution}
    IDFT 的定义为
    \begin{align*}
        x(n) = \frac{1}{N}\sum_{k = 0}^{N - 1}X(k)W_N^{-kn},
    \end{align*}
    可以将其分为两组,
    \begin{align*}
        x(n) & = \frac{1}{N}\sum_{k = 0}^{N/2 - 1}X(2k)W_N^{-2kn}
            + \frac{1}{N}\sum_{k = 0}^{N/2 - 1}X(2k + 1)W_N^{-(2k + 1)n} \\
        & = \frac{1}{N}\sum_{k = 0}^{N/2 - 1}G(k)W_{N/2}^{-kn}
            + \frac{W_N^{-n}}{N}\sum_{k = 0}^{N/2 - 1}H(k)W_{N/2}^{-kn}.
    \end{align*}
    由于 $G(k) = X(2k), H(k) = X(2k + 1)$,其中 $k = 0, 1, \cdots, N / 2 - 1$,故
    \begin{align*}
        \begin{cases}
            \sum_{k = 0}^{N/2 - 1}X(2k)W_{N/2}^{-kn}
                = \sum_{k = 0}^{N/2 - 1}G(r)W_{N/2}^{-kn} = \frac{N}{2}g(n), \\
            \sum_{k = 0}^{N/2 - 1}X(2k + 1)W_{N/2}^{-kn}
                = \sum_{k = 0}^{N/2 - 1}H(r)W_{N/2}^{-kn} = \frac{N}{2}h(n),
        \end{cases}
    \end{align*}
    其中 $n = 0, 1, \cdots, N/2 - 1$。因此
    \begin{align*}
        x(n) = \frac{1}{2}\left(g(n) + W_N^{-n}h(n)\right), \quad n = 0, 1, \cdots, N / 2 - 1.
    \end{align*}
    再考虑后半部分,
    \begin{align*}
        x\left(\frac{N}{2} + n\right) & = \frac{1}{N}\sum_{k = 0}^{N/2 - 1}X(2k)W_{N/2}^{-k\left(\frac{N}{2} + n\right)}
            + \frac{W_N^{-\left(\frac{N}{2} + n\right)}}{N}\sum_{k = 0}^{N/2 - 1}X(2k + 1)W_{N/2}^{-k\left(\frac{N}{2} + n\right)} \\
        & = \frac{1}{N}\sum_{k = 0}^{N/2 - 1}X(2k)W_{N/2}^{-kn}
            - \frac{W_N^{-n}}{N}\sum_{k = 0}^{N/2 - 1}X(2k + 1)W_{N/2}^{-kn} \\
        & = \frac{1}{N}\sum_{k = 0}^{N/2 - 1}G(k)W_{N/2}^{-kn}
            - \frac{W_N^{-n}}{N}\sum_{k = 0}^{N/2 - 1}H(k)W_{N/2}^{-kn} \\
        & = \frac{1}{2}\left(g(n) - W_N^{-n}h(n)\right), \quad n = 0, 1, \cdots, N / 2 - 1.
    \end{align*}
    综上所述,
    \begin{align*}
        x(n) = \frac{1}{2}\left(g(n) + W_N^{-n}h(n)\right),
            \quad x\left(\frac{N}{2} + n\right) = \frac{1}{2}\left(g(n) - W_N^{-n}h(n)\right),
    \end{align*}
    其中 $n = 0, 1, \cdots, N / 2 - 1$。
\end{solution}
