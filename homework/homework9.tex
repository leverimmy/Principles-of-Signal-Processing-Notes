\subsection{作业 9}

\begin{homework}
    已知序列 $x(n)$ 的长度为 $N$,$x(n) = x_r(n) + \mathi x_i(n)$,
    其中 $x_r(n)$ 和 $x_i(n)$ 分别是 $x(n)$ 的实部和虚部。
    设 $x(n)$ 的 $N$ 点 DFT 为 $X(k)$,令 $X(k) = X_{ep}(k) + X_{op}(k)$,
    其中 $X_{ep}(k)$ 为共轭对称序列,$X_{op}(k)$ 为共轭反对称序列,即
    \begin{align*}
        X_{ep}(k) & = X_{ep}^*(N-k), \quad k = 0, 1, \cdots, N-1, \\
        X_{op}(k) & = -X_{op}^*(N-k), \quad k = 0, 1, \cdots, N-1.
    \end{align*}
    \begin{enumerate}[label=(\arabic*)]
        \item 试用序列 $X(k)$ 分别表示序列 $X_{ep}(k)$ 和 $X_{op}(k)$。
        \item 证明:$\DFT{x_r(n)} = X_{ep}(k), \DFT{\mathi x_i(n)} = X_{op}(k)$,其中 DFT 点数均为 $N$。
    \end{enumerate}
\end{homework}

\begin{homework}
    设有限长序列 $x(n)$ 的取值范围为 $0, 1, \cdots, N-1$,长度 $N$ 为偶数。
    若该序列的 $N$ 点 DFT 为 $X(k)$,试用 $X(k)$ 表示下列各序列的 DFT。
    \begin{enumerate}[label=(\arabic*)]
        \item 将 $x(n)$ 以 $N$ 为周期进行周期延拓,然后对 $0 \sim MN-1$ 点
            组成的有限序列求其 $MN$ 点 DFT。
        \item 将 $x(n)$ 按如下方式进行时域扩展,得到 $MN$ 点新序列 $y(n)$,
            求其 $MN$ 点 DFT。
            \begin{align*}
                y(n) = \begin{cases}
                    x(n/M), & n/M \in \set{Z}, \\
                    0, & n/M \notin \set{Z}.
                \end{cases}
            \end{align*}
        \item 在 $x(n)$ 尾部补上若干零,成为长度为 $MN$ 的有限长序列 $y(n)$,
            求其 $MN$ 点 DFT。
            \begin{align*}
                y(n) = \begin{cases}
                    x(n), & 0 \le n \le N - 1, \\
                    0, & N \le n \le MN - 1.
                \end{cases}
            \end{align*}
    \end{enumerate}
\end{homework}
