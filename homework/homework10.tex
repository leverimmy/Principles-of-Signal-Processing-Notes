\subsection{作业 10}

\begin{homework}
    已知某系统的差分方程如下式:
    \begin{align*}
        y(n) = x(n) + x(n - 3) + 0.7y(n - 1) + 0.6y(n - 2).
    \end{align*}
    \begin{enumerate}[label=(\arabic*)]
        \item 判断系统的脉冲响应类型。
        \item 画出该系统的信号流图。
    \end{enumerate}
\end{homework}

\begin{solution}
    \begin{enumerate}[label=(\arabic*)]
        \item 该系统的脉冲响应类型为无限脉冲响应。
        \item 该系统的直接 I 型信号流图如图 \ref{fig:signal_flow_diagram} 所示。
        \begin{figure}[H]
            \centering
            \tikzstyle{block} = [draw, rectangle, minimum height=1cm, minimum width=1cm]
            \tikzstyle{circ} = [draw, fill, circle, inner sep=1.5pt]
            \tikzstyle{no-circ} = [draw, circle, inner sep=0pt]
            \tikzstyle{sum} = [draw, circle]
            \tikzstyle{line} = [draw, -latex]
            \tikzstyle{no-arrow-line} = [draw, -]
            \tikzstyle{gainx} = [draw, isosceles triangle, isosceles triangle apex angle=60]
            \tikzstyle{gainy} = [draw, isosceles triangle, isosceles triangle apex angle=60, shape border rotate=180]
            \begin{tikzpicture}
                \node (input) {$x(n)$};
                \node [circ, right of=input, xshift=1cm] (circx) {};
                \path [no-arrow-line] (input) -- (circx);
                \node [sum, right of=circx, xshift=4cm] (sum) {$+$};
                \node [circ, right of=sum, xshift=4cm] (circy) {};
                \path [no-arrow-line] (sum) -- (circy);
                \node [right of=circy, xshift=1cm] (output) {$y(n)$};
                \path [line] (circy) -- (output);
        
                \node [block, below of=circx, yshift=-1cm] (zx1) {$Z^{-1}$};
                \path [line] (circx) -- (zx1);
                \node [block, below of=zx1, yshift=-2cm] (zx2) {$Z^{-1}$};
                \path [line] (zx1) -- (zx2);
                \node [block, below of=zx2, yshift=-2cm] (zx3) {$Z^{-1}$};
                \path [line] (zx2) -- (zx3);
        
                \node [block, below of=circy, yshift=-1cm] (zy1) {$Z^{-1}$};
                \path [line] (circy) -- (zy1);
                \node [block, below of=zy1, yshift=-2cm] (zy2) {$Z^{-1}$};
                \path [line] (zy1) -- (zy2);
        
                \path [line] (circx) -- (sum);
                \coordinate (zgx3) at ([xshift=2cm, yshift=-1cm] zx3);
                \path [no-arrow-line] (zx3) |- (zgx3);
                \path [line] (zgx3) -- (sum);
                \node [gainy, below of=zy1, xshift=-2cm, yshift=-0.5cm] (zgy1) {$0.7$};
                \node [circ, below of=zy1, yshift=-0.5cm] (circy1) {};
                \path [line] (circy1) -- (zgy1);
                \path [line] (zgy1.west) -- (sum);
                \node [gainy, below of=zy2, xshift=-2cm, yshift=-0.5cm] (zgy2) {$0.6$};
                \path [line] (zy2) |- (zgy2);
                \path [line] (zgy2.west) -- (sum);
            \end{tikzpicture}
            \caption{作业 \thehomework~ 的信号流图}
            \label{fig:signal_flow_diagram}
        \end{figure}
    \end{enumerate}
\end{solution}

\begin{homework}
    观察傅里叶变换(FT)、傅里叶级数(FS)、离散时间傅里叶变换(DTFT)、
    离散傅里叶变换(DFT)时域和频域的周期性和离散性,
    总结时域离散性和频域周期性的关系、时域周期性和频域离散性的关系。
\end{homework}

\begin{solution}
    时域离散,则频域周期;时域周期,则频域离散。
    反之,频域离散,则时域周期;频域周期,则时域离散。
\end{solution}
