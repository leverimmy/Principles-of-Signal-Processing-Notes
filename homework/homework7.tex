\subsection{作业 7}

\begin{homework}
    已知 $x(n)$ 的 DTFT 为 $X(\omega)$,试求下列各序列的 DTFT:
    \begin{enumerate}[label=(\arabic*)]
        \item $x(n) * x^*(-n)$
        \item $x(2n + 1)$
        \item $x(n) - x(n + 2)$
        \item $x(n) * x(n + 1)$
    \end{enumerate}
\end{homework}

\begin{homework}
    证明性质 \ref{property:DTFT-time-expand}。即,若 $X(\omega)$ 是 $x(n)$ 的 DTFT,则
    \begin{align*}
        y(n) = \begin{cases}
            x(n / L), & n = 0, \pm L, \pm 2L, \cdots, \\
            0, & \text{otherwise},
        \end{cases}
    \end{align*}
    的 DTFT $Y(\omega)$ 满足 $Y(\omega) = X(L \omega)$。
\end{homework}

\begin{homework}
    已知序列 $x(n)$ 和 $y(n)$ 的 DTFT 分别为 $X(\omega)$ 和 $Y(\omega)$。
    对周期函数 $X(\omega)$ 和 $Y(\omega)$,定义相关系数
    \begin{align*}
        R_{XY}(\omega) = \int_{-\pi}^{\pi}X(\omega' + \omega)Y^*(\omega')\D{\omega'}
            = \int_{-\pi}^{\pi}X(\omega')Y^*(\omega' - \omega)\D{\omega'}.
    \end{align*}
    记 $R_{XY}(\omega)$ 的 IDTFT 为 $r_{xy}(n)$,
    试用 $x(n)$ 和 $y(n)$ 表示 $r_{xy}(n)$。
\end{homework}
