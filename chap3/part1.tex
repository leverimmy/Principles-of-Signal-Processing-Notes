\subsection{面临的矛盾与问题}

\subsubsection{矛盾 1:如何在信号信息不受损的情况下对时域信号进行离散化?}

我们可以考虑对信号在时域上进行理想采样,也就是说,采样过程需要满足采样定理,
则可以保证在信息不受损的情况下对信号进行离散化:

\begin{enumerate}[label=(\arabic*)]
    \item 信号频带宽度有限,例如限制在 $[-\omega_M, \omega_M]$ 内。
    \item 采样频率满足 $\omega_s \ge 2\omega_M$。
\end{enumerate}

至此,矛盾 1 已解决。

\subsubsection{矛盾 2:如何从采样信号的频谱恢复出原信号的频谱?}

我们可以在时域上进行带限内插,从采样信号中恢复出原信号。
理论上只要在采样时满足采样定理,即可恢复出原信号的频谱。
但实际工程中,采样频率必须满足 $\omega_s > 2\omega_M$,
即\bd{严格大于}采样定理中的条件,才能保证内插还原的信号与原信号一致。

至此,矛盾 2 已解决。

\subsubsection{矛盾 3:如何计算采样信号的频谱?}

对于连续信号,我们可以直接使用傅里叶变换计算其频谱。
但对于\bd{没有解析式}的信号(例如某些实际观测到的信号),
无法通过解析式得到信号的连续频谱。

采样信号是离散的,我们该如何计算出采样信号的频谱呢?

\subsubsection{矛盾 4:在截取某一段信号进行频谱分析时,信号的频谱与原频谱相比会发生什么变化?}

在实际工程中,我们往往只能获取信号的有限长片段,而不是整个信号,
这是因为信号的持续时间往往会超出设备的处理能力。

\begin{enumerate}[label=(\arabic*)]
    \item 截取信号会不会影响对信号的分析?
    \item 截取信号得到的频谱与原频谱相比会发生什么变化?
\end{enumerate}

\subsubsection{矛盾 5:如何存储离散信号的频谱?如何从有限的离散频谱中恢复出采样信号?}

离散信号的频谱是连续的周期函数,所以我们可以
\begin{enumerate}[label=(\arabic*)]
    \item 存储有限长度的频谱(有限范围),可以只保存一个周期的频谱信息。
    \item 存储有限数目的频谱(有限数目),可以只存离散频率处的频谱值。
\end{enumerate}

后者本质上是对连续周期频谱中的某一周期的采样。由于函数在一个周期内必然是带限的,
因此如果按照满足采样定理的条件来采样,则可以通过插值方法恢复出原离散信号。

\subsubsection{矛盾 6:如何实现离散信号与离散频谱之间的快速计算?}

如果按照定义来实现离散信号的离散时间傅里叶变换,那么计算量将会非常大,
难以推广到实际应用中。因此,我们需要寻找一种快速计算的方法。

这包含两个问题:
\begin{enumerate}[label=(\arabic*)]
    \item 如何由离散信号计算出离散频谱?
    \item 如何由离散频谱计算出离散信号?
\end{enumerate}
