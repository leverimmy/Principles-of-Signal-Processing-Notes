\section*{前言}

本讲义是 2024 年秋季学期清华大学计算机系《信号处理原理》课程的讲义,
由\bd{贾珈老师}\footnote{\url{https://hcsi.cs.tsinghua.edu.cn/jiajia}}授课。

这份讲义涵盖了贾珈老师在课堂上讲授的\bd{全部课程内容},并在此基础上进行了适当的扩充和整理。
讲义中还包含了\bd{三次习题课}的题目及其解答,可供同学们作为期末考试的复习资料。

讲义主要由\bd{熊泽恩}\footnote{\url{https://github.com/leverimmy}}整理而成。
在整理的过程中,\bd{朱泓舟}\footnote{\url{https://github.com/zhuhz22}}同学指出了许多错误和不足之处,
提出了许多宝贵的意见和建议。\bd{黄宸宇}\footnote{\url{https://github.com/Ever727}}同学是我的室友,在整理过程中与我讨论了许多问题,
帮助我更好地理解课程内容。还有\bd{黄柏升}\footnote{\url{https://github.com/bflyer}}同学、\bd{王浩然}\footnote{\url{https://github.com/UbeCc}}同学、\bd{梁梓宸}\footnote{\url{https://github.com/lzcsam}}同学
和\bd{施程予}\footnote{\url{https://github.com/shch-y}}同学,他们也指出了讲义中的一些错误。

我们几位同学在整理过程中,反复研读课程 PPT,查阅大量相关资料,力求使讲义内容准确无误、条理清晰。
尽管我们已经尽了最大努力,但由于时间和水平有限,讲义中难免存在一些不足之处。
我们真诚地希望同学们在使用过程中,能够提出宝贵的意见和建议,帮助我们不断完善讲义内容。

同学们如果发现讲义中的错误,或者有任何疑问,
可以通过 GitHub 仓库上的 Issue 功能\footnote{\url{https://github.com/leverimmy/Principles-of-Signal-Processing-Notes/issues}}
向我们反馈。我们会尽快进行修改和回复。

《信号处理原理》作为一门计算机系专业主修课程,对于同学们今后的学习和研究具有重要意义。
我们希望通过这份讲义,能够为同学们的学习提供一些帮助,让大家更好地理解和掌握课程内容。
最后,我们衷心感谢贾珈老师的辛勤付出,以及所有支持和帮助我们的助教和同学。

希望这份讲义能够成为大家学习道路上的一盏明灯,照亮前行的道路。

\begin{flushright}
    \today
\end{flushright}
